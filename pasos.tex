\documentclass{beamer}
\usepackage{amsmath}
\usepackage{amssymb}

\title{Diseños Lógicos}
\author{Johanel, Fabrizio, Jeaustin}
\institute{Tecnológico de Costa Rica}
\date{Semestre I de 2023}


\begin{document}
\begin{frame}
\frametitle{Recepción de datos}
Se recibe la cantidad de bits junto con las variables asociadas a sus respectivos valores.
\begin{itemize}
\item $$\textit{bits} $ = $ \textit{6}$$
\item $$\textit{a} $ = $ \textit{d21}$$
\item $$\textit{b} $ = $ \textit{h10}$$
\end{itemize}
\note{Notas}
\end{frame}
\begin{frame}
\frametitle{Convertir datos a binario}
Se convierten los datos a listas de 0s y 1s para representar un valor binario.
\begin{itemize}
\item $$\textit{bits} $ = $ \textit{6}$$
\item $$\textit{a} $ = $ \textit{$\textit{} $ + $ \textit{[1, 0, 1, 0, 1]}$}$$
\item $$\textit{b} $ = $ \textit{$\textit{} $ + $ \textit{[1, 0, 0, 0, 0]}$}$$
\end{itemize}
\note{Notas}
\end{frame}

\begin{frame}
\maketitle
\note{Notas}
\end{frame}
\end{document}