\documentclass{beamer}
\usepackage{amsmath}
\usepackage{amssymb}

\title{Diseños Lógicos}
\author{Johanel, Fabrizio, Jeaustin}
\institute{Tecnológico de Costa Rica}
\date{Semestre I de 2023}


\begin{document}
\begin{frame}
\frametitle{Recepción de datos}
Se recibe la cantidad de bits junto con las variables asociadas a sus respectivos valores.
\begin{itemize}
\item $\text{bits}=\text{7}$
\item $\text{ooga}=\text{d26}$
\item $\text{booga}=\text{b11}$
\end{itemize}
\note{Notas}
\end{frame}
\begin{frame}
\frametitle{Convertir datos a binario}
Se convierten los datos a listas de 0s y 1s para representar un valor binario.
\begin{itemize}
\item $\text{ooga}=\text{}+\text{[0, 0, 1, 1, 0, 1, 0]}$
\item $\text{booga}=\text{}+\text{[0, 0, 0, 0, 0, 1, 1]}$
\end{itemize}
\note{Notas}
\end{frame}
\begin{frame}
\frametitle{Tomar el valor absoluto de los números}
Se toma el valor absoluto de los números para realizar la multiplicación.
\begin{itemize}
\item $\text{abs(ooga)}=\text{[0, 0, 1, 1, 0, 1, 0]}$
\item $\text{abs(booga)}=\text{[0, 0, 0, 0, 0, 1, 1]}$
\end{itemize}
\note{Notas}
\end{frame}
\begin{frame}
\frametitle{Multiplicación binaria}
Se realiza la multiplicación binaria (de valor absoluto) de los dos números binarios.
\begin{itemize}
\item $\text{abs(ooga)}\times\text{abs(booga)}=\text{[0, 0, 1, 1, 0, 1, 0]}\times\text{[0, 0, 0, 0, 0, 1, 1]}=\text{...}$
\end{itemize}
\note{Notas}
\end{frame}
\begin{frame}
\frametitle{Inicializar registro y empezar a multiplicar}
Inicializar el resultado como una lista de ceros con longitud $\text{2}\times\text{7}$ y empezar a recorrer los bits de abs(booga).
\begin{itemize}
\item $\text{resultado}=\text{[0, 0, 0, 0, 0, 0, 0, 0, 0, 0, 0, 0, 0, 0]}$
\end{itemize}
\note{Notas}
\end{frame}
\begin{frame}
\frametitle{Sumar o no sumar No.1}
Si el bit de $\text{abs(booga)}=\text{1}$, sumar $\text{abs(ooga)}\times\text{2}^{\text{0}}$ al resultado.
\begin{itemize}
\item $\text{abs(booga)[6]}=\text{1}\Longrightarrow\text{Sí se hace la suma}$
\end{itemize}
\note{Notas}
\end{frame}
\begin{frame}
\frametitle{Sumar No.1}
Sumar $\text{abs(ooga)}\times\text{2}^{\text{0}}$ al resultado.
\begin{itemize}
\item $\text{[0, 0, 1, 1, 0, 1, 0]}\ll\text{0}=\text{[0, 0, 0, 0, 0, 0, 0, 0, 0, 1, 1, 0, 1, 0]}$
\item $\text{resultado}+\text{producto}=\text{[0, 0, 0, 0, 0, 0, 0, 0, 0, 1, 1, 0, 1, 0]}$
\end{itemize}
\note{Notas}
\end{frame}
\begin{frame}
\frametitle{Sumar o no sumar No.2}
Si el bit de $\text{abs(booga)}=\text{1}$, sumar $\text{abs(ooga)}\times\text{2}^{\text{1}}$ al resultado.
\begin{itemize}
\item $\text{abs(booga)[5]}=\text{1}\Longrightarrow\text{Sí se hace la suma}$
\end{itemize}
\note{Notas}
\end{frame}
\begin{frame}
\frametitle{Sumar No.2}
Sumar $\text{abs(ooga)}\times\text{2}^{\text{1}}$ al resultado.
\begin{itemize}
\item $\text{[0, 0, 1, 1, 0, 1, 0]}\ll\text{1}=\text{[0, 0, 0, 0, 0, 0, 0, 0, 1, 1, 0, 1, 0, 0]}$
\item $\text{resultado}+\text{producto}=\text{[0, 0, 0, 0, 0, 0, 0, 1, 0, 0, 1, 1, 1, 0]}$
\end{itemize}
\note{Notas}
\end{frame}
\begin{frame}
\frametitle{Sumar o no sumar No.3}
Si el bit de $\text{abs(booga)}=\text{1}$, sumar $\text{abs(ooga)}\times\text{2}^{\text{2}}$ al resultado.
\begin{itemize}
\item $\text{abs(booga)[4]}=\text{0}\Longrightarrow\text{No se hace la suma}$
\end{itemize}
\note{Notas}
\end{frame}
\begin{frame}
\frametitle{Sumar o no sumar No.4}
Si el bit de $\text{abs(booga)}=\text{1}$, sumar $\text{abs(ooga)}\times\text{2}^{\text{3}}$ al resultado.
\begin{itemize}
\item $\text{abs(booga)[3]}=\text{0}\Longrightarrow\text{No se hace la suma}$
\end{itemize}
\note{Notas}
\end{frame}
\begin{frame}
\frametitle{Sumar o no sumar No.5}
Si el bit de $\text{abs(booga)}=\text{1}$, sumar $\text{abs(ooga)}\times\text{2}^{\text{4}}$ al resultado.
\begin{itemize}
\item $\text{abs(booga)[2]}=\text{0}\Longrightarrow\text{No se hace la suma}$
\end{itemize}
\note{Notas}
\end{frame}
\begin{frame}
\frametitle{Sumar o no sumar No.6}
Si el bit de $\text{abs(booga)}=\text{1}$, sumar $\text{abs(ooga)}\times\text{2}^{\text{5}}$ al resultado.
\begin{itemize}
\item $\text{abs(booga)[1]}=\text{0}\Longrightarrow\text{No se hace la suma}$
\end{itemize}
\note{Notas}
\end{frame}
\begin{frame}
\frametitle{Sumar o no sumar No.7}
Si el bit de $\text{abs(booga)}=\text{1}$, sumar $\text{abs(ooga)}\times\text{2}^{\text{6}}$ al resultado.
\begin{itemize}
\item $\text{abs(booga)[0]}=\text{0}\Longrightarrow\text{No se hace la suma}$
\end{itemize}
\note{Notas}
\end{frame}
\begin{frame}
\frametitle{Recortar resultado}
Recortar el resultado para la cantidad de bits en cuestión.
\begin{itemize}
\item $\text{[0, 0, 0, 0, 0, 0, 0, 1, 0, 0, 1, 1, 1, 0]}=\text{[1, 0, 0, 1, 1, 1, 0]}$
\end{itemize}
\note{Notas}
\end{frame}
\begin{frame}
\frametitle{Aplicando negativos}
Se determina el signo del resultado y se convierte a complemento a dos si es negativo.
\begin{itemize}
\item $\text{}+\text{[1, 0, 0, 1, 1, 1, 0]}\Longrightarrow\text{[1, 0, 0, 1, 1, 1, 0]}$
\end{itemize}
\note{Notas}
\end{frame}
\begin{frame}
\frametitle{Resultado}
Se muestra el resultado de la multiplicación binaria.
\begin{itemize}
\item $\text{Resultado}=\text{ooga}\times\text{booga}=\text{}+\text{[0, 0, 1, 1, 0, 1, 0]}\times\text{}+\text{[0, 0, 0, 0, 0, 1, 1]}=\text{[1, 0, 0, 1, 1, 1, 0]}$
\end{itemize}
\note{Notas}
\end{frame}

\begin{frame}
\maketitle
\note{Notas}
\end{frame}
\end{document}