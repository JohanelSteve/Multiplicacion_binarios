\documentclass{beamer}
\usepackage{amsmath}
\usepackage{amssymb}

\title{Diseños Lógicos}
\author{Johanel, Fabrizio, Jeaustin}
\institute{Tecnológico de Costa Rica}
\date{Semestre I de 2023}


\begin{document}
\begin{frame}
\frametitle{Recepción de datos}
Se recibe la cantidad de bits junto con las variables asociadas a sus respectivos valores.
\begin{itemize}
\item $\text{bits} = \text{6}$
\item $\text{a} = \text{ds-21}$
\item $\text{b} = \text{h10}$
\end{itemize}
\note{Notas}
\end{frame}
\begin{frame}
\frametitle{Convertir datos a binario}
Se convierten los datos a listas de 0s y 1s para representar un valor binario.
\begin{itemize}
\item $\text{bits} = \text{6}$
\item $\text{a} = \text{} - [0, 1, 0, 1, 0, 1]$
\item $\text{b} = \text{} + [0, 1, 0, 0, 0, 0]$
\end{itemize}
\note{Notas}
\end{frame}
\begin{frame}
\frametitle{Tomar el valor absoluto de los números}
Se toma el valor absoluto de los números para realizar la multiplicación.
\begin{itemize}
\item $\text{abs(a)} = [0, 1, 0, 1, 0, 1]$
\item $\text{abs(b)} = [0, 1, 0, 0, 0, 0]$
\end{itemize}
\note{Notas}
\end{frame}

\begin{frame}
\maketitle
\note{Notas}
\end{frame}
\end{document}