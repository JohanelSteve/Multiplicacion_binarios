\documentclass{beamer}
\usepackage{amsmath}
\usepackage{amssymb}

\title{Diseños Lógicos}
\author{Johanel, Fabrizio, Jeaustin}
\institute{Tecnológico de Costa Rica}
\date{Semestre I de 2023}


\begin{document}
\begin{frame}
\frametitle{Recepción de datos}
Se recibe la cantidad de bits junto con las variables asociadas a sus respectivos valores.
\begin{itemize}
\item $\text{bits} = \text{6}$
\item $\text{a} = \text{10}$
\item $\text{b} = \text{3}$
\end{itemize}
\note{Notas}
\end{frame}
\begin{frame}
\frametitle{Convertir datos a binario}
Se convierten los datos a listas de 0s y 1s para representar un valor binario.
\begin{itemize}
\item $\text{bits} = \text{6}$
\item $\text{a} = \text{} + [0, 0, 1, 0, 1, 0]$
\item $\text{b} = \text{} + [0, 0, 0, 0, 1, 1]$
\end{itemize}
\note{Notas}
\end{frame}
\begin{frame}
\frametitle{Tomar el valor absoluto de los números}
Se toma el valor absoluto de los números para realizar la multiplicación.
\begin{itemize}
\item $\text{abs(a)} = [0, 0, 1, 0, 1, 0]$
\item $\text{abs(b)} = [0, 0, 0, 0, 1, 1]$
\end{itemize}
\note{Notas}
\end{frame}
\begin{frame}
\frametitle{Multiplicación binaria}
Se realiza la multiplicación binaria (de valor absoluto) de los dos números binarios.
\begin{itemize}
\item $\text{abs(a)} \times \text{abs(b)} = \text{[0, 0, 1, 0, 1, 0]} \times \text{[0, 0, 0, 0, 1, 1]} = \text{...}$
\end{itemize}
\note{Notas}
\end{frame}
\begin{frame}
\frametitle{Inicializar registro}
Inicializar el resultado como una lista de ceros con longitud 2*self.bits.
\begin{itemize}
\item $\text{resultado} = [0, 0, 0, 0, 0, 0, 0, 0, 0, 0, 0, 0]$
\end{itemize}
\note{Notas}
\end{frame}
\begin{frame}
\frametitle{Recortar resultado}
Recortar el resultado para la cantidad de bits en cuestión.
\begin{itemize}
\item $[0, 0, 0, 0, 0, 0, 0, 1, 1, 1, 1, 0] = [0, 1, 1, 1, 1, 0]$
\end{itemize}
\note{Notas}
\end{frame}
\begin{frame}
\frametitle{Aplicando negativos}
Se determina el signo del resultado y se convierte a complemento a dos si es negativo.
\begin{itemize}
\item $\text{} + [0, 1, 1, 1, 1, 0] \Longrightarrow [0, 1, 1, 1, 1, 0]$
\end{itemize}
\note{Notas}
\end{frame}

\begin{frame}
\maketitle
\note{Notas}
\end{frame}
\end{document}