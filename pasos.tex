\documentclass{beamer}
\usepackage{amsmath}
\usepackage{amssymb}
\usepackage[utf8]{inputenc}

\title{Diseños Lógicos}
\author{Johanel, Fabrizio, Jeaustin}
\institute{Tecnológico de Costa Rica}
\date{Semestre I de 2023}


\begin{document}
\begin{frame}
\frametitle{Recepción de datos}
Se recibe la cantidad de bits junto con las variables asociadas a sus respectivos valores.
\begin{itemize}
\item $\texttt{bits}=\texttt{3}$
\item $\texttt{b}=\texttt{s-10}$
\item $\texttt{a}=\texttt{hsf}$
\end{itemize}
\note{Notas}
\end{frame}
\begin{frame}
\frametitle{Convertir datos a binario}
Se convierten los datos a listas de 0s y 1s para representar un valor binario.
\begin{itemize}
\item $\texttt{b}=\texttt{}-\texttt{010}$
\item $\texttt{a}=\texttt{}+\texttt{111}$
\end{itemize}
\note{Notas}
\end{frame}
\begin{frame}
\frametitle{Tomar el valor absoluto de los números}
Se toma el valor absoluto de los números para realizar la multiplicación.
\begin{itemize}
\item $\texttt{abs(b)}=\texttt{010}$
\item $\texttt{abs(a)}=\texttt{111}$
\end{itemize}
\note{Notas}
\end{frame}
\begin{frame}
\frametitle{Multiplicación binaria}
Se realiza la multiplicación binaria (de valor absoluto) de los dos números binarios.
\begin{itemize}
\item $\texttt{abs(b)}\times\texttt{abs(a)}=\texttt{010}\times\texttt{111}=\texttt{...}$
\end{itemize}
\note{Notas}
\end{frame}
\begin{frame}
\frametitle{Procedimientos}

\begin{itemize}
\item $\texttt{\ \ \ }\texttt{010}$
\item $\texttt{\ \ x}\texttt{111}$
\item $\text{------------}\text{}$
\item $\texttt{\ \ \ }\texttt{010}$
\item $\texttt{\ \ }\texttt{010}$
\item $\texttt{+}\texttt{010}$
\item $\text{------------}\text{}$
\item $\texttt{}\texttt{001110}$
\end{itemize}
\note{Notas}
\end{frame}
\begin{frame}
\frametitle{Recortar resultado}
Recortar el resultado para la cantidad de bits en cuestión.
\begin{itemize}
\item $\texttt{001110}=\texttt{110}$
\end{itemize}
\note{Notas}
\end{frame}
\begin{frame}
\frametitle{Aplicando negativos}
Se determina el signo del resultado y se convierte a complemento a dos si es negativo.
\begin{itemize}
\item $\texttt{}-\texttt{110}\Longrightarrow\texttt{010}$
\end{itemize}
\note{Notas}
\end{frame}
\begin{frame}
\frametitle{Resultado}
Se muestra el resultado de la multiplicación binaria.
\begin{itemize}
\item $\texttt{Resultado}=\texttt{b}\times\texttt{a}=\texttt{}-\texttt{010}\times\texttt{}+\texttt{111}=\texttt{010}$
\end{itemize}
\note{Notas}
\end{frame}

\begin{frame}
\maketitle
\note{Notas}
\end{frame}
\end{document}