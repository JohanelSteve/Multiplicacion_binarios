\documentclass{beamer}
\usepackage{amsmath}
\usepackage{amssymb}

\title{Diseños Lógicos}
\author{Johanel, Fabrizio, Jeaustin}
\institute{Tecnológico de Costa Rica}
\date{Semestre I de 2023}


\begin{document}
\begin{frame}
\frametitle{Recepción de datos}
Se recibe la cantidad de bits junto con las variables asociadas a sus respectivos valores.
\begin{itemize}
\item $\text{bits} = \text{6}$
\item $\text{a} = \text{s-10}$
\item $\text{b} = \text{3}$
\end{itemize}
\note{Notas}
\end{frame}
\begin{frame}
\frametitle{Convertir datos a binario}
Se convierten los datos a listas de 0s y 1s para representar un valor binario.
\begin{itemize}
\item $\text{a} = \text{} - \text{[0, 0, 1, 0, 1, 0]}$
\item $\text{b} = \text{} + \text{[0, 0, 0, 0, 1, 1]}$
\end{itemize}
\note{Notas}
\end{frame}
\begin{frame}
\frametitle{Tomar el valor absoluto de los números}
Se toma el valor absoluto de los números para realizar la multiplicación.
\begin{itemize}
\item $\text{abs(a)} = \text{[0, 0, 1, 0, 1, 0]}$
\item $\text{abs(b)} = \text{[0, 0, 0, 0, 1, 1]}$
\end{itemize}
\note{Notas}
\end{frame}
\begin{frame}
\frametitle{Multiplicación binaria}
Se realiza la multiplicación binaria (de valor absoluto) de los dos números binarios.
\begin{itemize}
\item $\text{abs(a)} \times \text{abs(b)} = \text{[0, 0, 1, 0, 1, 0]} \times \text{[0, 0, 0, 0, 1, 1]} = \text{...}$
\end{itemize}
\note{Notas}
\end{frame}
\begin{frame}
\frametitle{Inicializar registro y empezar a multiplicar}
Inicializar el resultado como una lista de ceros con longitud 2 $\times$ self.bits y empezar a recorrer los bits de abs(b).
\begin{itemize}
\item $\text{resultado} = \text{[0, 0, 0, 0, 0, 0, 0, 0, 0, 0, 0, 0]}$
\end{itemize}
\note{Notas}
\end{frame}
\begin{frame}
\frametitle{Sumar o no sumar No.1}
Si el bit de abs(b) es 1, sumar abs(a)$\times 2^{i}$.
\begin{itemize}
\item $\text{abs(b)[5]} = \text{1} \Longrightarrow \text{Sí se hace la suma}$
\end{itemize}
\note{Notas}
\end{frame}
\begin{frame}
\frametitle{Sumar No.1}
Sumar abs(a)$\times 2^{i}$ al resultado.
\begin{itemize}
\item $\text{[0, 0, 1, 0, 1, 0]} \ll \text{0} = \text{[0, 0, 0, 0, 0, 0, 0, 0, 1, 0, 1, 0]}$
\item $\text{resultado} + \text{producto} = \text{[0, 0, 0, 0, 0, 0, 0, 0, 1, 0, 1, 0]}$
\end{itemize}
\note{Notas}
\end{frame}
\begin{frame}
\frametitle{Sumar o no sumar No.2}
Si el bit de abs(b) es 1, sumar abs(a)$\times 2^{i}$.
\begin{itemize}
\item $\text{abs(b)[4]} = \text{1} \Longrightarrow \text{Sí se hace la suma}$
\end{itemize}
\note{Notas}
\end{frame}
\begin{frame}
\frametitle{Sumar No.2}
Sumar abs(a)$\times 2^{i}$ al resultado.
\begin{itemize}
\item $\text{[0, 0, 1, 0, 1, 0]} \ll \text{1} = \text{[0, 0, 0, 0, 0, 0, 0, 1, 0, 1, 0, 0]}$
\item $\text{resultado} + \text{producto} = \text{[0, 0, 0, 0, 0, 0, 0, 1, 1, 1, 1, 0]}$
\end{itemize}
\note{Notas}
\end{frame}
\begin{frame}
\frametitle{Sumar o no sumar No.3}
Si el bit de abs(b) es 1, sumar abs(a)$\times 2^{i}$.
\begin{itemize}
\item $\text{abs(b)[3]} = \text{0} \Longrightarrow \text{No se hace la suma}$
\end{itemize}
\note{Notas}
\end{frame}
\begin{frame}
\frametitle{Sumar o no sumar No.4}
Si el bit de abs(b) es 1, sumar abs(a)$\times 2^{i}$.
\begin{itemize}
\item $\text{abs(b)[2]} = \text{0} \Longrightarrow \text{No se hace la suma}$
\end{itemize}
\note{Notas}
\end{frame}
\begin{frame}
\frametitle{Sumar o no sumar No.5}
Si el bit de abs(b) es 1, sumar abs(a)$\times 2^{i}$.
\begin{itemize}
\item $\text{abs(b)[1]} = \text{0} \Longrightarrow \text{No se hace la suma}$
\end{itemize}
\note{Notas}
\end{frame}
\begin{frame}
\frametitle{Sumar o no sumar No.6}
Si el bit de abs(b) es 1, sumar abs(a)$\times 2^{i}$.
\begin{itemize}
\item $\text{abs(b)[0]} = \text{0} \Longrightarrow \text{No se hace la suma}$
\end{itemize}
\note{Notas}
\end{frame}
\begin{frame}
\frametitle{Recortar resultado}
Recortar el resultado para la cantidad de bits en cuestión.
\begin{itemize}
\item $\text{[0, 0, 0, 0, 0, 0, 0, 1, 1, 1, 1, 0]} = \text{[0, 1, 1, 1, 1, 0]}$
\end{itemize}
\note{Notas}
\end{frame}
\begin{frame}
\frametitle{Aplicando negativos}
Se determina el signo del resultado y se convierte a complemento a dos si es negativo.
\begin{itemize}
\item $\text{} - \text{[0, 1, 1, 1, 1, 0]} \Longrightarrow \text{[1, 0, 0, 0, 1, 0]}$
\end{itemize}
\note{Notas}
\end{frame}

\begin{frame}
\maketitle
\note{Notas}
\end{frame}
\end{document}